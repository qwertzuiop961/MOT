% !TEX encoding = UTF-8 Unicode
\documentclass[12pt,a4paper]{article}
\usepackage[ngerman]{babel}
\usepackage[utf8]{inputenc}
\usepackage[T1]{fontenc}
\usepackage{lmodern}
\usepackage{graphicx}
\usepackage{enumerate}
\usepackage[]{epstopdf}
\usepackage[margin=1in]{geometry}
\usepackage{titling}
\usepackage{hyperref}
\usepackage[german]{cleveref}
\usepackage{amsmath,amsthm,verbatim,amssymb,amsfonts,amscd}
\renewcommand{\familydefault}{\sfdefault}
\usepackage[miktex]{gnuplottex}
\usepackage[decimalsymbol=comma,separate-uncertainty=true,expproduct=\cdot,
            uncertainty-separator=\pm]{siunitx}
\setlength{\droptitle}{-2cm}
\setlength{\parindent}{0pt}
\title{Auswertung zu IQ13:\\
       magneto-optische Falle\\
       WiSe 15/16\\
       Block III}
\author{Ramin Javadi 2993630 und Felix Schrader 3053850}
\date{}
\begin{document}
\maketitle
\tableofcontents
\pagebreak
\section{Theorie einer MOT}
  \subsection{Optische Melasse}
  \subsection{Magnetfeld}
  \subsection{Übergänge von ${}^{87}$Rb}
\section{Bestandteile der MOT}
  \subsection{Die Laser}
    In dem Versuchsaufbau werden zwei Halbleiterlaser (Kühl- und Rückpumplaser)
    verwendet. Der Kühllaser wird mit ca. \SI{85}{\mA}, der Rückpumper mit ca.
    \SI{160}{\mA} betrieben. Durch Anlegen eines Stromes in Durchlassrichtung
    einer p-n-Diode entsteht eine dünne Schicht, in der es eine Besetzungsinversion
    mit Löchern im Valenz- und Elektronen im Leitungsband gibt. Als Resonator
    dienen die Stirnflächen der Diode. Durch Änderung des Stromes kann man die
    emittierte Wellenlänge ändern, allerdings erhöht sich gleichzeitig die
    Laserleistung. Ein Problem der Laser ist, dass sich durch leichte
    Temperaturänderungen die optische Weglänge des Lasers ändert und es somit zu
    Modensprüngen kommt. Um diese zu umgehen (und zum feineren Einstellen der
    Wellenlänge) wird ein Gitter als externe Cavity verwendet. Die erste
    Beugungsordnung wird dabei zurückreflektiert. Der Winkel, unter dem der Strahl
    von Gitter reflektiert wird ist von der Wellenlänge abhängig, daher kann man
    durch drehen des Gitters eine Änderung der Wellenlänge erreichen. Um das Gitter
    zu drehen ist es auf einem Piezo-Kristall befestigt. Durch Änderung des Stromes
    am Piezo-Element, ändert sich dessen Temperatur und damit dessen Ausdehnung, das
    Gitter wird also bewegt. Die Reflexion in die nullte Ordnung wird als Laserlicht
    verwendet. Um Rückreflexe in die Diode zu verhindern steht hinter den Lasern
    jeweils ein Isolator.
  \subsection{Laserlocken}
    Um die Laser zu koppeln wird ein kleiner Teil des Lichts in den
    Spektroskopiepfad ausgekoppelt. Dort befindet sich ein Aufbau zur
    Sättigungsspektroskopie. Dabei wird der Strahl durch eine Glasscheibe geschickt
    (es entstehen zwei Reflexe, einer an der Vorder- einer an der Rückseite der
    Scheibe) und über einen Spiegel in eine Rubidiumzelle geleitet (Pumpstrahl). Die
    beiden Reflexe von der Glasplatte werden ebenfalls in die Rubidiumzelle geschickt,
    allerdings von der anderen Seite (dadurch sind sie jeweils in die andere Richtung
    dopplerverschoben, wie der Pumpstrahl). Dabei passiert der stärkere Reflex
    (Abfragestrahl) in der Zelle einen Bereich, den auch der Pumpstrahl durchläuft.
    Beide brennen dadurch Löcher in das normale Dopplerprofil, die im gleichen
    Abstand links und rechts vom Maximum sind. Stellt man die Wellenlänge richtig ein,
    können sich beide Löcher im Profil überlagern (die Rubidiumatome können also von
    beiden Strahlen angeregt werden). Dies kann passieren, wenn man genau auf
    Resonanz mit einem Übergang ist (da beide Strahlen um den gleichen Betrag
    dopplerverschoben sind ist Absorption von Licht aus beiden Strahlen gleich
    wahrscheinlich $\rightarrow$ Lamb-Dip) oder wenn die Wellenlänge genau zwischen
    zwei Übergängen liegt (beide Strahlen regen jeweils unterschiedliche Übergänge an
    $\rightarrow$ Cross-Over-Resonanz). Da auf dem Oszilloskop die Absorption des
    Laserlichts angezeigt wird, erscheinen diese Dips als kleine Maxima im
    Dopplerprofil. Der schwächere Reflex von der Glasplatte durchläuft die
    Rubidiumzelle an einer anderen Stelle und misst somit zum Vergleich das reine
    Dopplerprofil.
    
    Um die Laser zu locken wird der Strom am Piezoelement so eingestellt, dass man
    genau den gewünschten Lamb-Dip erhält. Dabei wird auf dem Oszilloskop das
    Signal vom Abfragestrahl (mit Dopplerprofil) angezeigt, zum Locken wird die
    Differenz von Abfragestrahl und Dopplerprofil verwendet. Leichte Änderungen der
    Wellenlänge (z.B. durch wackeln der Cavity oder Temperaturänderungen) sorgen
    dafür, dass das Signal nicht mehr auf einem Maximum ist. Dann wird der Strom
    automatisch auf den richtigen Wert geregelt um wieder das Maximum zu erreichen.
    Da allerdings nicht klar ist, auf welcher Seite vom Maximum man sich befindet
    wird das Signal zunächst abgeleitet. Die Ableitung des Signals erfolgt durch
    Modulierung des Stromes mit einer Sinuskurve. Dies erlaubt Messungen auf beiden
    Seiten leicht neben dem betrachteten Punkt und somit eine Abschätzung der Steigung
    also eine Ableitung. Das Locken erfolgt dann auf einem Nulldurchgang der Ableitung.
  \subsection{Der AOM}
    
  \subsection{Der TA}
  \subsection{Die Vakuumkammer}
  \subsection{Die Kamera}
\section{Messungen}
  \subsection{Intensitätsmessungen im Strahlengang}
  \subsection{Vermessung der Strahlbreite}
  \subsection{Eichung der Kamera}
\end{document}